\documentclass[11pt]{article}
\usepackage{a4wide}
\usepackage[usenames]{color}
\usepackage[authoryear,round]{natbib}
\bibliographystyle{plainnat}
\pagestyle{plain}

\begin{document}

% Title

\begin{center}
{\Large \textbf{Working with the {DICOM} and {NIfTI} Data Standards in \textsf{R}}}

\bigskip

March~8, 2011

\end{center}

\section*{Associate Editor}

\subsection*{General Comments}

\begin{itemize}
  
\item Generating the pdf document from Sweave includes information on
  different machines (where the creation has been done). Thus the
  \texttt{audit.trail} in Fig.10 should be edit slightly in the final
  version (working directory).
  
  We're happy to do this on the final version of the manuscript.
  
\end{itemize}

\subsection*{Specific Comments}

\begin{itemize}
  
\item Can the grey scale contrast be enhanced in the images?

  We have modified the \texttt{zlim} in Figures~3, 6 and~9.

\end{itemize}

\subsection*{JSS-specific Formatting}

\begin{itemize}
  
\item \texttt{section} and \texttt{subsection} should be formatted in
  sentence style.
  
  Changed.

\end{itemize}

\section*{Reviewer \#1}

We would like to thank the reviewer for all his/her comments and
suggestions.

\subsection*{Main Remarks}

\begin{itemize}

\item The paper clearly describes what the packages can do. However,
  some questions raised my mind while reading the paper. How should
  you describe the users that are going the use the packages? And how
  did you implement their specific needs?  

\item For example 1: Suppose I am a practical researcher, which
  acquires imaging data to support a research question.  Which
  functions would mostly provide in my needs?

  We have included a table that lists the major functions available in
  both the \textbf{oro.dicom} and \textbf{oro.nifti} packages.

\item For example 2: Suppose I am a statistician that wants to use
  your package to enhance the statistical data analysis of imaging
  data.  What are the main functionalities of header manipulation that
  I can use?

  See our answer to the previous question.

\item Above questions are more or less addressed in the paper, but I
  think it should be better specified what the specific purposes of
  the packages are.  I would like to suggest a description of the
  intended users in the introduction of the paper.  Additionally, some
  clear examples that are specifically intended for the described
  users would be highly beneficial for the clearness of the paper.

  Thank-you for your questions and recommendations.  We have added
  text in the introduction to address who should want to use the
  packages and how they might want to use the packages.  In addition
  we have included a fuctional MRI (fMRI) example to illustrate how
  basic \textsf{R} functions can be applied to medical imaging data,
  once loaded into \textsf{R} using \textbf{oro.nifti}.

\end{itemize}

\subsection*{Minor Remarks}

\begin{itemize} 

\item section 2, 1st paragraph: DICOM abbreviation is already defined
  in the introduction

  This text has been removed.

\item section2, 2nd paragraph: At the end you say that the fourth
  column of the DICOM format is not implemented. Why is this the case?

  The table of information was taken directly from the DICOM
  documentation.  To the best of our knowledge there is nothing of
  value in the fourth column of Table~1.

\item subsection 2.2: last sentence: extractHeader

  Changed.

\end{itemize}

\section*{Reviewer \#2}

We would like to thank the reviewer for all his/her comments and
suggestions.

\subsection*{Major Points}

\begin{itemize}

\item This brings me directly to the main criticism regarding this
  manuscript: There is no application of a statistical test or a
  linear modeling on imaging data.  Instead, pre-calculated maps are
  loaded in chapter 3.7 to present overlay capabilities.  One may ask
  why these libraries were just added to R?  Furthermore, the authors
  stated in chapter 3.6, that (interactive) visualization is ``however
  best performed outside of R''.  Again: Where is then the benefit of
  the library, when these things can be done outside of R more
  fluidly?
  I therefore strongly suggest adding at least a (real world)
  statistical example with image data, where R can show its
  outstanding performance in its field.  This could convince the
  reader from the benefit they would have when using this library in
  R. The increased space requirements for that can then be saved at
  the expense of the detailed description of some (binary) internals
  of DICOM and ANALYZE/NIfTI.

  The main purpose of the \textbf{oro.dicom} and \textbf{oro.nifti}
  packages is to read and write NIfTI, ANALYZE and DICOM images, to 
  manipulate data formats and to convert DICOM to NIfTI. From this, 
  any statistical procedure in \textbf{R} can be used to analyze the
  images. As example, we added a section on the analysis of a functional MRI 
  data set in Section~3.7.  

\end{itemize}

\subsection*{Minor Points}

\begin{itemize}

\item Page 2, Chapter 2, line 1: The first sentence is (somehow)
  redundant.

  Re-worded.

\item Page 2, chapter 2, line 4: There are two two-byte (or 16 bit
  integer) sequences, not ``two four-byte sequences'' as mentioned,
  which form a tag.

  Changed.

\item Page 2, chapter 2, line 7: Please, use ``pixel data'' instead of
  ``data'' enclosed in quotation marks. Besides that, this tag is not
  necessarily found (but surely most often) only at the end. There
  could be e.g. two of them: One with the data itself (at the end) and
  another one containing a thumbnail and embedded in a sequence
  (somewhere in the header).

  Changed.

\item Page 2, chapter 2, line 12: The value representation string is
  only available when explicit encoding is used. Please reword.

  Re-worded.

\item Page 2, chapter 2, line 14 and Table 1: Please use ``Unlimited''
  or similar instead of ``0''.

  Added text in the table and body to emphasize this point.

\item Page 5, chapter 2.1: The authors should clearly point out here,
  which kinds of DICOM objects can be accessed. I suppose that only
  monochrome images with 16 bit depth are supported. But the DICOM
  standard defines a lot more (RGB-colorized, JPEG- or
  JPEG2000-packed, .).

  Text has been added.

\item Page 5, chapter 2.1, line 10: DICOM headers are encoded
  explicitly or implicitly, which is stated in the appropriate
  transfer syntax in the meta header.  A strategy as ``If the
  character string in bytes four and five do not correspond .'' sounds
  to me a little bit too heuristic, but can accepted, if the meta
  header is not available..

  You make an interesting point, but we are not familiar with the meta
  header field that signifies ``explicit'' or ``implicit'' encoding.
  Could you please provide the (group,element) values for this field?

\item Page 8, chapter 2.2, line 9: Typo: Write ``extractHeader''
  instead of ``extratHeader''.

  Changed.

\item Page 9, chapter 2.2, line 7: Because in MRI slices are not
  acquired at once, they also cannot have the same repetition time
  (TR). I suggest to change it to ``. in a single repetition time (TR)
  frame .''.

  Changed.

\item Page 10, chapter 3, line 4: Analyze 7.5 might be the last known
  version of this file format, but ``most recent'' sounds like
  ``actual'', which is not true here: For Mayo Clinics, the developer
  of the software ANALYZE, it is an obsolete format, superseded by
  AnalyzeAVW, which is much more flexible. It has survived in the
  scientific community for practical reasons and by early adoptions by
  SPM. Please, use another wording here, like ``last available'' or
  similar.

  Text has been added and modified.

\item Page 12, chapter 3.2, line 1: The voxel offset is already known
  for Analyze 7.5. It was possible (but only rarely used) to start
  with image information not from the beginning of the .img-file.

  Text at the end of the first paragraph has been modifed to include
  the case of ``pairs of files'' that occur under the ANALYZE and
  split NIfTI formats.

\item Page 13, chapter 3.2, line 3 and page 14, chapter 3.3, line 8:
  This is not right. Following the radiological convention the
  radiologist looks into the brain from below, because the patient or
  volunteer is usually scanned lying head first and in supine
  position, and the feet point towards the radiologist. From this
  point of view, left and right are flipped. The neurological
  convention instead is oriented on the anatomy ``as is'' and uses
  therefore not a mirrored visualization.

  Thank-you for pointing this out, it appears that we have reversed our
  definitions.  This has been corrected.

\item Page 17, chapter 3.5, line 4: What does ``ecode = 6'' mean?

  Text has been added.

\item Page 19, chapter 3.7, line 6 and Figure 7: The center of the
  crosshair does not point at the thalamus in the proper sense,
  because the thalamus is a paired structure and the main functional
  groups are situated left and right from the displayed mid-sagittal
  position.

  Modified.

\item Page 22, chapter 4.1, line 5: ``list(dcm =, datatype = 4, .'' is
  truncated after ``DIM =''

  We have not been able to find a way to impose line breaks in the
  \texttt{dput} command.  We will most likely have to edit the final
  version of the manuscript to produce visually-pleasing output.

\item Page 27: There is no URL for Mr. Thornton.

  This line has been removed from the manuscript.

\end{itemize}

%--------------------------------------------------------------------%

\bibliography{dicom_nifti}

%--------------------------------------------------------------------%

\end{document}
